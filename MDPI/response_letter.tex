\documentclass[12pt]{dinbrief}
%\usepackage{english}
\usepackage[english]{babel}
\usepackage[utf8]{inputenc}
\usepackage{color}
\usepackage[textsize=small]{todonotes}

\usepackage[unicode=true, pdfusetitle, bookmarks=true, bookmarksnumbered=false, bookmarksopen=false, breaklinks=false, pdfborder={0 0 0}, pdfborderstyle={}, backref=false, colorlinks=true] {hyperref}

%\usepackage[pdfborder={0 0 0}]{hyperref}


\date{\today}

\place{Prague}

\nowindowtics
\nowindowrules

\newcommand{\theeditor}{Dr. David Díaz-Díaz}

\newcommand{\thejournal}{\textit{Gels }}
\newcommand{\cl}{\mathrm{Cl^-}}
\newcommand{\na}{\mathrm{Na^+}}

%\newcommand{\ie}{\textit{i.~e.} } 
\enabledraftstandard
\newcommand{\ie}{\textit{i.~e.} }
\newcommand{\cref}{c^{\ominus}}
%\newcommand{\cs}{c_{s}}
\newcommand{\kT}{k_\mathrm{B}T}
\newcommand{\kB}{k_\mathrm{B}}
\newcommand{\lb}{l_\mathrm{B}}
%\newcommand{\NA}{N_{\mathrm{A^-}}}
\newcommand{\NA}{N_{\mathrm{A}}}
\newcommand{\muna}{\mu_\mathrm{Na^+}}

\newcommand{\mucl}{\mu_\mathrm{Cl^-}}
\newcommand{\muca}{\mu_\mathrm{Ca^{2+}}}
\newcommand{\muh}{\mu_\mathrm{H^+}}
\newcommand{\mua}{\mu_\mathrm{A^-}}
\newcommand{\muha}{\mu_\mathrm{HA}}
\newcommand{\muoh}{\mu_\mathrm{OH}}

\newcommand{\cna}{c_\mathrm{Na^+}}
\newcommand{\ccl}{c_\mathrm{Cl^-}}
\newcommand{\cca}{c_\mathrm{Ca^{2+}}}
\newcommand{\ch}{c_\mathrm{H^+}}
\newcommand{\cp}{c_\mathrm{p}}
\newcommand{\nna}{n_\mathrm{Na^+}}
\newcommand{\ncl}{n_\mathrm{Cl^-}}
\newcommand{\Nna}{N_\mathrm{Na^+}}
\newcommand{\Ncl}{N_\mathrm{Cl^-}}

\newcommand{\ncleq}{\widetilde{N}_\mathrm{Cl^-}}
\newcommand{\nca}{N_\mathrm{Ca^{2+}}}

\newcommand{\superin}{^\mathrm{in}}
\newcommand{\subin}{_\mathrm{in}}
\newcommand{\subi}{_\mathrm{i}}

\newcommand{\gel}{^\mathrm{gel}}
\newcommand{\tot}{^\mathrm{tot}}
\newcommand{\out}{^{\mathrm{out}}}
\newcommand{\coh}{c_\mathrm{OH}}
\newcommand{\bulk}{^{\mathrm{b}}}
\renewcommand{\H}{\mathrm{H^+}}
\newcommand{\A}{\mathrm{A^-}}
\newcommand{\AH}{\mathrm{AH}}
%%%%%%%%%%%%%%%%%%%%%%%%%%%%%% LyX specific LaTeX commands.
%% A simple dot to overcome graphicx limitations
\newcommand{\lyxdot}{.}
\newcommand{\todoi}[1]{\todo[inline]{#1}}
\newcommand{\br}{\mathrm{Br^-}}
\newcommand{\h}{\mathrm{H^+}}
\newcommand{\ka}{\mathrm{K^+}}
\newcommand{\oh}{\mathrm{OH^-}}
\newcommand{\ca}{\mathrm{Ca^{2+}}}
\newcommand{\mg}{\mathrm{Mg^{2+}}}
\newcommand{\so}{\mathrm{SO_4^{2-}}}

\newcommand{\EI}{E_{\mathrm{I}}}
\newcommand{\EII}{E_{\mathrm{II}}}
\newcommand{\SI}{S_{\mathrm{I}}}
\newcommand{\SII}{S_{\mathrm{II}}}
\newcommand{\NI}{N_{\mathrm{I}}}
\newcommand{\NII}{N_{\mathrm{II}}}
\newcommand{\VI}{V_{\mathrm{I}}}
\newcommand{\VII}{V_{\mathrm{II}}}
\newcommand{\FI}{F_{\mathrm{I}}}
\newcommand{\FII}{F_{\mathrm{II}}}
\newcommand{\nnaI}{N^{\na}_{\mathrm{I}}}
\newcommand{\nclI}{N^{\cl}_{\mathrm{I}}}
\newcommand{\nnaII}{N^{\na}_{\mathrm{II}}}
\newcommand{\nclII}{N^{\cl}_{\mathrm{II}}}

\newcommand{\Ka}{K_{\mathrm{A}}}
\newcommand{\pKa}{\mathrm{p}\Ka}
\newcommand{\pK}{\mathrm{p}K}
\newcommand{\pH}{\mathrm{pH}}
\newcommand{\mol}{\mathrm{mol}}
\newcommand{\molperl}{\mathrm{mol/l}}
\newcommand{\kg}{\mathrm{kg}}
\newcommand{\res}{^{\mathrm{res}}}
\newcommand{\pHres}{\pH\res}
\newcommand{\pHgel}{\pH\gel}
\newcommand{\cs}{c_{\mathrm{s}}}
\newcommand{\csres}{\cs\res}
\newcommand{\Vgel}{V_\mathrm{gel}}
\newcommand{\Ngel}{N_\mathrm{gel}}
\newcommand{\Pgel}{\Pi}
\newcommand{\Pres}{P_\mathrm{res}}
\newcommand{\Pout}{P_\mathrm{out}}
\newcommand{\Vout}{V_\mathrm{out}}
\newcommand{\Vbox}{V_0}
\newcommand{\PE}{polyelectrolyte{}}




\begin{document}

\begin{letter}{
Dr. Oleg Rud\\
Department of Physical and Macromolecular Chemistry\\
Faculty of Science, Charles University\\
Hlavova 8, 128 43 Prague, Czech Republic\\
E-mail: oleg.rud@natur.cuni.cz\\
}
\vspace*{-1.5cm}

\opening{Dear \theeditor,}


In response to your e-mail from 23d August 2022, we hereby submit the revised
version, of our manuscript, entitled "Water desalination using polyelectrolyte hydrogel. Gibbs ensemble modeling''.

We carefully considered all reviewer comments and introduced changes in the manuscript in order to address them.
Below, we provide a point-by-point response to all reviewer comments, including a brief description of the modifications introduced in response to these comments.
Additionally, we provide a pdf version of the manuscript with the relevant changes highlighted in colour.

Main changes in the manuscript include:
\begin{itemize}
    \item 
    \item 
    \item 
    \item 
\end{itemize}

We believe that, after considering the modifications in response to reviewer comments, you will find our manuscript suitable for publication in Desalination.

On behalf of all authors,
with best regards\\
\\
Oleg V. Rud\\

%\closing{ \ 
%}

\end{letter}

% modify the page to use the space more efficiently
\addtolength{\textwidth}{1.5cm}
\addtolength{\textheight}{2.0cm}



\textbf{Reviewer 1}
\textit{}

\begin{enumerate}
\item \textit{The main findings of your results should be highlighted in abstract} 
We have rewritten the abstract. The abstract now contains main findings of our paper.

\item \textit{Please do not separate introduction in three sections combine with problem and technology you used and challenges and your solutions.}

\item \textit{In page 4 heading 2. (How to start with and ??)}.\\
Corrected.

\item \textit{All formula and models should be supported with reference for instant equation 3.}
Added todonote

\item \textit{In page 6, please remove the footer, it is not a book to describe issues below}.\\
Corrected. Now this sentence is appears in text (in brackets).


\item \textit{How do you think control factors in open system such as pressure and temperature}

\end{enumerate}


\textbf{Reviewer 2}
\textit{This manuscript is about creating a new model the thermodynamic equilibrium between coexisting phases of the gel and supernate aqueous solution.}
\begin{enumerate}

\item \textit{Abstract needs to re-write as and please follow the IMRAD instruction.}
We have rewritten the abstract. We believe that the abstract is much more attractive now.
\item \textit{There is introduction in the abstract, however, it needs to write more about the aim of this work, method, result and conclusion.}
We have rewritten the abstract according to your advice.
\item \textit{Avoid to use word 'we' in the abstract as well as manuscript.}
Corrected.
\item \textit{line 247 in what table?}
We meant Table 1. Corrected.
\item \textit{Where is the method of this work?} 
The method idea is outlined in the paragraph "Physics behind the desalination process" and described in the section "Theory behind the simulation". We did not want to bother the readers too much with technical aspects of the smulations, so technical details are in ESI.
\item \textit{It is suggested that to carefully improve this manuscript as well as the writing and grammatical error.}
We improved the writing style of the manuscript and corrected the grammatical errors.
\item \textit{Please explain more detail about the process, how the process run and how the data could be explained and become more interesting for reader.}

\end{enumerate}


\textbf{Reviewer 3}

\textit{}
\begin{enumerate}
\item \textit{The full stop in the title should be revised.}
Noted. Revised.
\item \textit{The abstract needs to be revised and be more attractive. The research gap, novelty and main findings should be concisely mentioned in the abstract.}
We have rewritten the abstract. We believe that the abstract is much more attractive now.
\item \textit{In this current shape, the manuscript looks like a report. The authors have to revise the flow of the context to attract the readers.}
We have revised and changed the manuscript to be more attractive for the readers. 
\item \textit{The manuscript must be improved by checking once again in English and technical writing. }
We improved the writing style of the manuscript and corrected the grammatical errors.

\item \textit{The whole manuscript must be supported by enough recent references. Revise the whole context.}
\item \textit{The beginning of the introduction part must be revised to be more attractive.}
\item \textit{The authors can follow the following references to generally highlight the global contamination and environmental pollution problems at the beginning of the introduction part  \href{https://doi.org/10.1016/j.jwpe.2022.102847}{https://doi.org/10.1016/j.jwpe.2022.102847} ; \href{https://doi.org/10.1016/j.ceramint.2022.05.151}{https://doi.org/10.1016/j.ceramint.2022.05.151} ; 
\href{https://doi.org/10.3390/catal12050500}{https://doi.org/10.3390/catal12050500} ; 
\href{https://doi.org/10.1007/s11356-022-21160-7}{https://doi.org/10.1007/s11356-022-21160-7} ; 
\href{https://doi.org/10.3390/ma15134547}{https://doi.org/10.3390/ma15134547}. 
Then, highlight the water desalination technologies and Gibbs ensemble (\href{https://doi.org/10.3390/nano10020293}{https://doi.org/10.3390/nano10020293}).}
\item \textit{Authors should indicate a clear gap in knowledge which this study seeks to bridge, and potentially contribute to knowledge. }

\item \textit{L 125-126, meaningless sentence. Revise the whole context.}
Corrected.
\item \textit{All sections/subsections titles must be revised to be expressive.}
\item \textit{Statistical physics considerations including setup and calculations should be added.}
We added the paragraph about Gibbs ensemble.
\item \textit{The authors should identify the limitations of this study and the recommended future studies in the conclusion part.
}
We identified the study limitations and considered the implications and future perspectives of the study. We added the respective paragraphs.
\end{enumerate}






%~ \bibliographystyle{unsrt}
%~ \bibliography{desalination}
\bibliographystyle{unsrt}
\bibliography{lit/gibbs}
\end{document}


