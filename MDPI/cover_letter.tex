\documentclass[12pt]{dinbrief}

%\usepackage{english}
\usepackage[english]{babel}
\usepackage[utf8]{inputenc}
\usepackage{color}
\usepackage[textsize=small]{todonotes}

\usepackage[unicode=true, pdfusetitle, bookmarks=true, bookmarksnumbered=false, bookmarksopen=false, breaklinks=false, pdfborder={0 0 0}, pdfborderstyle={}, backref=false, colorlinks=true] {hyperref}

%\usepackage[pdfborder={0 0 0}]{hyperref}


\date{\today}

\place{Prague}

\nowindowtics
\nowindowrules

\newcommand{\theeditor}{Dr. David Díaz-Díaz}

\newcommand{\thejournal}{\textit{Gels }}
\newcommand{\cl}{\mathrm{Cl^-}}
\newcommand{\na}{\mathrm{Na^+}}

\newcommand{\ie}{\textit{i.~e.} } 
\enabledraftstandard

\begin{document}

\begin{letter}{
Dr. Oleg Rud\\
Department of Physical and Macromolecular Chemistry\\
Faculty of Science, Charles University\\
Hlavova 8, 128 43 Prague, Czech Republic\\
E-mail: oleg.rud@natur.cuni.cz\\
}
\vspace*{-1.5cm}

\opening{Dear \theeditor,}

Herewith we would like to submit our manuscript: ``\textit{Water desalination using polyelectrolyte hydrogel. Gibbs ensemble modeling}''
authored by
Mikhail Laktionov, Lucie Nová and Oleg V. Rud,
 to be considered for publication in \thejournal. 
In the submitted manuscript we modeled hydrogel in Gibbs ensemble, \ie in equilibrium with a small amount of aqueous solution, and demonstrated that the compression of the gel in such an ensemble slightly decreases the salinity of the outside solution. 
We employed this phenomenon and modeled the process of water desalination as a cascade of hydrogel swellings and compressions changing the outside salinity from $\sim 0.1$ to 0.01 mol/l.

Our study develops a general understanding of the physics of polyelectrolyte hydrogels. It also brings new insights into their application to seawater desalination. Thus, we believe that the study fits within the scope of \thejournal and that it will be of interest to its broad readership.

\vspace{5ex}

We confirm that neither the manuscript nor any parts of its content are currently under consideration or published in another journal.

All authors have approved the manuscript and agree with its submission to (journal name).



Yours sincerely,\\
on behalf of all co-authors,

\hspace{80ex}\ O.V. Rud\\





%The advantage of the proposed desalination procedure over the forward osmosis is that the gel acts both as a draw solute and as a separation membrane. 
%That helps to avoid various blocking phenomena (characteristic of membrane processes), which required a regular chemical and mechanical cleaning.
%Moreover, the relatively low pressures needed for hydrogel compression allow to  maintain the hydrogel mechanical properties and integrity for its reuse in a serial cycle.


\end{letter}
\end{document}
